\documentclass{article}

\usepackage{color}
\usepackage{graphicx}
\usepackage{tabularx}
\usepackage[frenchb]{babel}
\usepackage[utf8]{inputenc}
\usepackage[T1]{fontenc}
\usepackage{lmodern}

\usepackage{geometry,wrapfig,lipsum}
 \geometry{
 top=20mm,
 bottom=20mm,
 }


\title{Hacking}
\author{Justal Kevin}
\date{30/10/2015}
\renewcommand{\contentsname}{Table des mati\`eres} 
 
\newcommand\invisiblesection[1]{%
  \refstepcounter{section}%
  \addcontentsline{toc}{section}{\protect\numberline{\thesection}#1}%
  \sectionmark{#1}} 
 
\begin{document}

\begin{center}
\textbf{\Huge{MANAGEMENT}}\\
\line(1,0){300}\\
Négociations\\
\vspace{3cm}
\includegraphics[width=\textwidth]{1}\\
\vspace{3cm}
\textbf{JUSTAL KEVIN}\\
2015-2016\\
\vspace{3cm}
\textbf{\color{blue}{\underline{justal.kevin@gmail.com}}}\\
\end{center}

\newpage
\tableofcontents

\newpage
\section{Présentation}
\hspace*{0.6cm}Permettez moi tout d'abord de me présenter, Je suis Monsieur Justal kévin, chargé des relations Entreprises du groupe Centaure. Nous sommes spécialisé dans la sécurité routière depuis de nombreuses années et participons activement à la baisse d'accident de la route. Via nos locaux ultra-moderne et nos équipes professionel, nous organisons très souvent des stages unique breveté et reconnu qui inculque aux conducteurs la volonté et les moyens pour ne pas avoir d'accident. Ce que nous recherchons avant tout est un comportement anti-accident. Actuellement 22\% des journées de travail sont perdue à cause d'accidents de la route. Plus que le drame humains, l'accident à un prix lourd pour toutes entreprises qui est souvent difficile à amortir. Il est donc important de tout faire pour les éviter. Notre entreprise avec un minumum de temps et à faible cout permettent de réduire le nombre d'accident de manière significatif. Nos statistiques montre que nous réduisons le nombre d'accident de 50\% après que les conducteurs est assistés à nos stages. Ces stages sont donc un moyen efficace et simple de diminuer le nombre de sinistre et les couts en assurance.
\vspace{0.2cm}\\
\section{Recherche des besoins}
Avant de rentrer dans les détails, j'aimerais vous posez quelques questions :
\vspace{0.2cm}\\
- De quelles manières, formez-vous les conducteurs de votre entreprise aux risques routiers ?\\
- N'est ce pas trop imposant comme formation ?
- Qui s'occupe actuellement de cette formation ? Et quand vos conducteurs passe-t-il cette formation ?\\
- Est-ce vous qui avez choisit cette entreprise pour les formations de vos conducteurs ?\\
- Etiez-vous seul ? Si non, qui prenait la décision en plus de vous ?\\
- Comment avez vous pris la décision ?\\
- Et quand étais-ce exactement ? \\
- Votre entreprise a-t-elle connue des accidents cette année ? Dans quelle fréquence ?\\
- Racontez-moi comment se déroule ces formations ?\\
- Sur quels critères avez-vous décidé de faire confiance à cette entreprise pour les formations de vos conducteurs ?\\
- Combien coute actuellement cette formation ?\\
- Que pensez-vous de cette formation ?\\
- Que souhaiteriez-vous de plus de cette entreprise ?\\
- Avez-vous consulté d'autres entreprises sur la sécurité routière ?
\vspace{0.2cm}\\



\end{document}